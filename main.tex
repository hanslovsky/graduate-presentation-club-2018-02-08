\documentclass[aspectratio=169]{beamer}

\usepackage{graphicx}
\usepackage{import}
\usepackage{media9}
\usepackage{multimedia}
\usepackage{pifont}% http://ctan.org/pkg/pifont
\newcommand{\cmark}{\ding{51}}%
\newcommand{\xmark}{\ding{55}}%

\usepackage{tikz}
\usetikzlibrary{3d}
\usetikzlibrary{backgrounds}
\usetikzlibrary{decorations.pathreplacing}
\usetikzlibrary{fit}
% \usetikzlibrary{external}
\usetikzlibrary{matrix}
\usetikzlibrary{positioning}
\usetikzlibrary{scopes} % brilliant library!

% \tikzexternalize

\pgfdeclarelayer{background}
\pgfdeclarelayer{foreground}
\pgfsetlayers{background,main,foreground}


\usetheme{Janelia}

\title[Janelia Theme]{The Metamorphosis of BigCAT --- Toward Streamlined Proofreading}
\subtitle{Toward Streamlined Proofreading}
% \subtitle{A custom modern minimalist Beamer theme designed from scratch}
\author{Philipp Hanslovsky}
\date{February 18, 2018}

\setcounter{showSlideNumbers}{1}

\begin{document}

\setcounter{showProgressBar}{0}
\setcounter{showSlideNumbers}{0}

\frame{\titlepage}

\begin{frame}
    \frametitle{Circuit Reconstruction}
    \includegraphics{fig/reconstruction-pipeline.pdf}
    tbd: reference for this figure
    % \begin{itemize}
    %       \item Dense reconstruction in EM data
    %       \item Connectivity graph
    %       \item Avoid critical errors
    % \end{itemize}
    % \begin{itemize}
    %       \item Machine Learning Algorithms
    %       \item Neuron boundary prediction
    %       \item Agglomeration
    %       \item Proof Reading
    % \end{itemize}
\end{frame}

\begin{frame}
    \frametitle{BigCAT}
    % \begin{itemize}
    %       \item Arbitrary Re-Slicing
    %       \item Painting (!!)
    %       \item Agglomeration
    %       \item Annotations
    %       \item Potential for heavy workload tasks on client
    % \end{itemize}

    \begin{tikzpicture}[overlay]
        \useasboundingbox (0,0) rectangle(\the\paperwidth,\the\paperheight);
        \coordinate (width) at (\the\paperwidth,\the\paperheight);
        \fill[black,anchor=west,inner sep=0] ($(0,0)-(width)$) rectangle (\the\paperwidth,\the\paperheight);
    \end{tikzpicture}%
    
    \def\factor{1.0}
    \movie[%
    width=\factor\textwidth,%
    height=\factor\textheight,%
    ]{\includegraphics[width=\factor\textwidth]{vid/bigcat-cremi-placeholder.jpg}}{vid/bigcat-cremi-b-aligned.mov}
\end{frame}

\begin{frame}
    \frametitle{BigCAT -- Want!}
    \begin{itemize}
          \item<1-> Ortho views with arbitrary reslicing
          \item<2-> 3D visualization of meshes
        \begin{itemize}
              \item<3-> Mesh generation on the fly for arbitrary data sets 
        \end{itemize}
          \item<4-> Multi-resolution painting
          \item<5-> (Guided) agglomeration with solver server
          \item<6-> Broadly applicable and extensible
        \begin{itemize}
              \item<7-> Heavy number crunching on client side
        \end{itemize}
    \end{itemize}
\end{frame} 

\begin{frame}
    \frametitle{Why not Neuroglancer}
    \begin{itemize}
          \item<1->[\color{green}\cmark] Ortho views with arbitrary reslicing
          \item<2->[\only<2>{\color{green}\cmark}\only<3->{\color{yellow}\cmark}] 3D visualization of meshes
        \begin{itemize}
              \item<3->[\color{red}\xmark] Mesh generation on the fly only for python in memory data sets
        \end{itemize}
          \item<4->[\color{red}\xmark] Multi-resolution painting
          \item<5->[\color{red}\xmark] (Guided) agglomeration with solver server
          \item<6->[\only<6>{\color{green}\cmark}\only<7->{\color{yellow}\cmark}] Broadly applicable and extensible
        \begin{itemize}
              \item<7->[\color{red}\xmark] Heavy number crunching on client side limited \\ \fbox{\includegraphics[width=4cm]{fig/chrome-heap-limit.png}}
        \end{itemize}
    \end{itemize}
% https://stackoverflow.com/a/23705446/1725687
% https://groups.google.com/a/chromium.org/forum/#!topic/chromium-dev/TUddM_rtgi4
\end{frame}

% \begin{frame}
%     \begin{tikzpicture}[overlay]
%         \useasboundingbox (0,0) rectangle(\the\paperwidth,\the\paperheight);
%         \coordinate (width) at (\the\paperwidth,\the\paperheight);
%         \fill[black,anchor=west,inner sep=0] ($(0,0)-(width)$) rectangle (\the\paperwidth,\the\paperheight);
%     \end{tikzpicture}%
% \end{frame}

\begin{frame}
    \frametitle{Agglomeration}

    \begin{tikzpicture}[overlay]
        \useasboundingbox (0,0) rectangle(\the\paperwidth,\the\paperheight);
        \coordinate (width) at (\the\paperwidth,\the\paperheight);
        \fill[black,anchor=west,inner sep=0] ($(0,0)-(width)$) rectangle (\the\paperwidth,\the\paperheight);
    \end{tikzpicture}%
    
    \def\factor{1.0}
    \movie[%
    width=\factor\textwidth,%
    height=\factor\textheight,%
    ]{}{vid/assignments.mp4}
    % ]{\includegraphics[width=\factor\textwidth]{vid/assignments-placeholder.jpg}}{vid/assignments.mp4}
    % show video
    % backend: c9ae08b
    % bigcat:  e84388e
    % start with fragment 2660 (blue)
\end{frame}

\begin{frame}
    \frametitle{Agglomeration}
    \subimport{fig/}{agglomeration}
\end{frame}

\begin{frame}
    \frametitle{Agglomeration -- Actions}
    \begin{itemize}
          \item Merge two fragments
          \item Split two fragments
          \item Confirm visible segment
          \item Confirm two visible segments
    \end{itemize}
\end{frame}

\begin{frame}
    \frametitle{Painting}

    \begin{tikzpicture}[overlay]
        \useasboundingbox (0,0) rectangle(\the\paperwidth,\the\paperheight);
        \coordinate (width) at (\the\paperwidth,\the\paperheight);
        \fill[black,anchor=west,inner sep=0] ($(0,0)-(width)$) rectangle (\the\paperwidth,\the\paperheight);
    \end{tikzpicture}%

    \def\factor{1.0}
    \movie[%
    width=\factor\textwidth,%
    height=\factor\textheight,%
    ]{}{vid/paint.mp4}
    % ]{\includegraphics[width=\factor\textwidth]{vid/paint-placeholder.jpg}}{vid/paint.mp4}
\end{frame}

    \def\colorOne{red}
    \def\colorTwo{blue}
    \def\colorThree{yellow}
    \def\colorFour{green}
    \def\colorFive{magenta}
    \def\cellSize{1.4em}
\begin{frame}
    \frametitle{Painting}
    \subimport{fig/}{paint}
\end{frame}

\begin{frame}
    \frametitle{Painting}
    \subimport{fig/}{paint-continued}
\end{frame}

\begin{frame}
    \frametitle{Mesh Generation}
    \begin{tikzpicture}[overlay]
        \useasboundingbox (0,0) rectangle(\the\paperwidth,\the\paperheight);
        \coordinate (width) at (\the\paperwidth,\the\paperheight);
        \fill[black,anchor=west,inner sep=0] ($(0,0)-(width)$) rectangle (\the\paperwidth,\the\paperheight);
    \end{tikzpicture}%

    \def\factor{1.0}
    \movie[%
    width=\factor\textwidth,%
    height=\factor\textheight,%
    ]{}{vid/mesh.mp4}
\end{frame}

\begin{frame}
    \frametitle{Mesh Generation}
    \centering
    \vspace{3em}
    \subimport{fig/}{mesh-cache}
\end{frame}

\begin{frame}
    \frametitle{Cool Hacks}
    \begin{tikzpicture}[overlay]
        \useasboundingbox (0,0) rectangle(\the\paperwidth,\the\paperheight);
        \coordinate (width) at (\the\paperwidth,\the\paperheight);
        \fill[black,anchor=west,inner sep=0] ($(0,0)-(width)$) rectangle (\the\paperwidth,\the\paperheight);
    \end{tikzpicture}%

    \def\factor{1.0}
    \movie[%
    width=\factor\textwidth,%
    height=\factor\textheight,%
    ]{}{vid/bigcat-synapses.mp4}
\end{frame}

\begin{frame}
    \frametitle{Name}
    tbd
    BigCAT may be known under a different name in the future:
    https://github.com/saalfeldlab/bigcat/issues/23
\end{frame}

\begin{frame}
    \frametitle{Links}
    \begin{itemize}
          \item[BigCAT] 
    \end{itemize}
\end{frame}

\begin{frame}
    \frametitle{Acknowledgements}
    tbd
    \begin{itemize}
          \item Saalfeld lab
          \item Tobias
          \item Jan, Vanessa
    \end{itemize}
\end{frame}

    



\end{document}


%%% Local Variables:
%%% mode: latex
%%% TeX-master: t
%%% End:
